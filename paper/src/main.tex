\documentclass[12pt]{article}

% Packages
\usepackage{sbc-template}
\usepackage{graphicx,url}
\usepackage[brazil]{babel}   
\usepackage[utf8]{inputenc}  
\usepackage{lipsum}

\sloppy

% Info
\title{Albi:\\ A gro compiler to the SBML standard}
\author{Alek Frohlich\inst{1}, Gustavo Biage\inst{1}}
\address{Departamento de Informatica e Estatística – Universidade Federal de Santa Catarina \\
Florianopolis – SC – Brazil
  \email{\{alek.frohlich,gustavo.c.biage\}@grad.ufsc.br}
}

\begin{document}
\maketitle

% Topics:
%       => The growth of the area leads to the development of multicellular devices
%       => Gro is well suited for this purpose
%       => However it isn't ready for compatibility
%       => Introduce the objective of making it compatible
\begin{abstract}


    The growth of research areas such as synthetic biology and systems biology leads to an increased willing to develop new, larger mathematical models to describe complex biological behavior. In order to enable natural flow of development of those models, scientists must have access to tools which increase the level of abstraction and enable reuse of biological components. Increased efforts are being put on solving these two problems. The present work tackles those problems by interfacing the existing programming language gro with SBML for model interchangeability and ease of model representation.


\end{abstract}

% Topics:
%       => What does albi fix that libSBML does not?
%       => What are the reasons we chose gro as albi's primary language?
%       => What are the disadvantages of the gro simulator
%       => What are the advantages of having the SBML model for a gro program?
%       => Prepare next sections
\section{Introduction}


    Even though a Systems Biology Markup Language API library (LibSBML) has already been developed \cite{Bornstein2008}, it only ought to be useful in cases where a new model is to be developed. In cases where there is a preexisting model, the availability of an SBML library doesn't help much since the previous model would have to be entirely rewritten to fit the API. Instead, the present work proposes a language parser that generates SBML code from previously built gro models. gro is a language for programming, modeling, specifying and simulating the behavior of cells in growing micro colonies of microorganisms \cite{Jang2012}. It has been made the primary source of the parser considering it has many interesting syntactical constructs such as rate statements, program definitions and bacterial instantiation which can be neatly represented, respectively, in SBML documents as reactions, local name spaces and compartments.
    
    As language support goes, gro is still tightly coupled with it's original simulator. Although the simulator has recently undergone big improvements in the sense that simulations behave more realistically \cite{Gutirrez2017}, it doesn't change the fact that using it is the only available way to validate gro models. This hinders the reproducibility of experiments built with the language. SBML models, on the other hand, are supported by more than 100 software systems \cite{Hucka2007}, making them more likely to be verified and reused.
    
    % syntactical analysis could be interpreted as a manual task by context
    If gro code were to be translated into SBML, the document could be further preserved as an entry in BioModels. BioModels is a database of curated SBML and CellML models maintained with the intent of providing researches with models related to a particular disease, biological process or molecular complex \cite{LeNovere2006}. Having a model curated is an even stronger guarantee of trustworthiness since the process is done manually for each model \cite{LeNovere2006}: Newly submitted models go through a series of syntactical analysis to confirm that they are valid syntactically and well formatted; The model is then numerically verified against the results claimed by the authoring paper; Finally, the validated model gets manually annotated and cross-linked to facilitate later search for it.

    % is it really repression of TFs?
   The rest of the paper is structured as follows: In Sect. 2, we introduce the language parser and show some examples of gro source code being translated into SBML documents; In Sect. 3, we explain how the libraries Tellurium and roadrunner were used to generate SBML documents; In Sect. 4, we compile a gro implementation of the repressilator, a synthetic oscillator based upon gene regulatory networks with negative feedback loops, to equivalent SBML and then simulate it on Copasi, a pathway simulator \cite{Hoops2006}; In Sect. 5, we summarize what's been done within this paper. 
    
    
% Topics:
%       => Which syntax does the parser recognize?
%       => Examples of generated SBML
\section{The parser}
    % The parser recognizes a subset of the gro syntax. It does so because the main multicellular features of gro aren't yet supported by SBML, a known single-cell modelling language. That i
    \lipsum[1]
    
\section{The Tellurium framework}
    \lipsum[1]

\section{Study case: The Repressilator}
    \lipsum[1]

\subsection{Mathematical Model}
    \lipsum[1]
    % Cite example
    \cite{Hucka2003}
    % Figure example
    (Figure~\ref{fig:repressilator}).

% Oscillation
\begin{figure}[ht]
\centering
\includegraphics[width=.5\textwidth]{repressilator.jpg}
\caption{Repressilator}
\label{fig:repressilator}
\end{figure}

\begin{center}
    \begin{figure}[h]
        
        \begin{subfigure}
            \includegraphics[scale = 0.4]{my_plot-inc.eps}
            \caption{Caption1}
            \label{fig:subim1}
        \end{subfigure}
        \begin{subfigure}
            \includegraphics[scale = 0.4]{my_plot-inc.eps}
            \caption{Caption1}
            \label{fig:subim1}
        \end{subfigure}
    \end{figure}
    % \setlength{\unitlength}{1pt}
    % \begin{picture}(0,0)
    % \includegraphics[scale = 0.4]{my_plot-inc}
    % \end{picture}%
    % \begin{picture}(576,432)(0,0)
    % \fontsize{10}{0}
    % \end{picture}
\end{center}

\subsection{Extracting behavior from Gro}
    \lipsum[1]
    
\subsection{Simulating output on Copasi}
    \lipsum[1]

\section{Conclusion}
    \lipsum[1]
    
% Topics:
%       => It would be nice if SBML were to be upgraded to fit multicellular models
%       => Then we could extend the parser's syntax to make gro the multicellular systems biology %              programming language
\section{Future works}
    As it is now, there isn't a multicellular systems biology programming language. The authors believe nonetheless, that gro could be that language, if only a model exchange format such as SBML were to handle it. After that's done, extending the parser's syntax even without the help of APIs in the likes of libSBML and Tellurium, would be a relatively simple task.

% References
\bibliographystyle{sbc}
\bibliography{sbc-template}

\end{document}
